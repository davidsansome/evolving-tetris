\documentclass[a4paper,12pt]{article}

\usepackage[pdftex]{graphicx}
\usepackage[pdftex,breaklinks=true,colorlinks=true,linkcolor=black,filecolor=blue,urlcolor=blue]{hyperref}
\usepackage{color}
\usepackage[paper=a4paper,hmargin=2cm,vmargin=3cm]{geometry}
\usepackage{lscape}

% Paragraph style
\setlength\parindent{0in}
\setlength\parskip{0.1in}

% URL style
\def\UrlFont{\small\tt}

%opening
\title{COMP6026 Assignment 2 \\
An Evolutionary Approach to Tetris}
\author{David Sansome $<$ds505$>$}

\begin{document}
\bibliographystyle{plain}

\maketitle

\begin{abstract}

This report describes the design, implementation and results of an evolutionary
algorithm that plays the popular game of Tetris \cite{AboutTetris}.
Tetris is a falling-block game where the aim is to survive as long as possible
by dropping blocks (\emph{Tetraminos}) into a fixed-size board.
A row is cleared from the board if it is completely filled, and the game ends
when there is no more room in the board for new blocks.
The challenge therefore is to place blocks in such a way as to minimize holes
in the board and to clear as many rows as possible.

The algorithm used is as described by Mandl et.\ al \cite{Mandl2005}.
At each step of the game the computer evaluates all possible subsequent boards
and chooses the best move based on a rating function.
This function is a weighted sum of various board attributes, such as the height
of filled cells and the number of holes in the board.
The algorithm evolves the weights used in this function and finds sets of
weights that lead to longer games of Tetris.

Our implementation uses fitness proportional selection and uniform crossover.

\end{abstract}

\tableofcontents

\section{Methods}

\subsection{Representation of Individuals}
\subsection{Fitness Evaluation}
\subsection{Generational}
\subsection{Selection}
\subsection{Crossover and Mutation}
\subsection{Implementation Details}

\section{Results}

\section{Further Work}

\bibliography{../COMP6026}

\end{document}
