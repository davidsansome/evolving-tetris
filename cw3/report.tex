\documentclass[a4paper,11pt]{article}

\usepackage[pdftex]{graphicx}
\usepackage[pdftex,breaklinks=true,colorlinks=true,linkcolor=black,filecolor=blue,urlcolor=blue]{hyperref}
\usepackage{color}
\usepackage[paper=a4paper,hmargin=3cm,vmargin=4cm]{geometry}
\usepackage{lscape}
\usepackage{amsmath}
\usepackage{listings}
\usepackage{fancyhdr}
\usepackage{subfig}
\usepackage{pdflscape}

% Paragraph style
\setlength\parindent{0in}
\setlength\parskip{0.1in}

% URL style
\def\UrlFont{\small\tt}

\lstset{breaklines=true,basicstyle=\small,tabsize=3,numbers=left,numberstyle=\tiny,numbersep=5pt,emptylines=1}

% Setup fancy headings - copied from lshort
\pagestyle{fancy}
% with this we ensure that the chapter and section
% headings are in lowercase.
\renewcommand{\sectionmark}[1]{%
        \markright{\thesection\ #1}}
\fancyhf{} % delete current header and footer
\fancyhead[LE,RO]{\bfseries\thepage}
\fancyhead[LO]{\bfseries\rightmark}
\fancyhead[RE]{\bfseries\leftmark}
\renewcommand{\headrulewidth}{0.5pt}
\renewcommand{\footrulewidth}{0pt}
\addtolength{\headheight}{0.5pt} % space for the rule
\fancypagestyle{plain}{%
   \fancyhead{} % get rid of headers on plain pages
   \renewcommand{\headrulewidth}{0pt} % and the line
}

\newcounter{savedcounter}

%opening
\title{COMP6026 Assignment 3 \\
Coevolving an ideal Tetris trainer}
\author{David Sansome $<$ds505$>$}

\begin{document}
\bibliographystyle{plain}

\maketitle

\begin{abstract}



\end{abstract}

\clearpage
\tableofcontents

\section{Technical Improvements}

\subsection{Tetris players}

The first part of this coursework described an algorithm to evolve a Tetris
player.
Each individual player in the population used a different board-rating function
to choose the best move to make at each stage of the game.
The player's decision was based on six criteria:

\begin{enumerate}
  \item \emph{Pile Height}: The total height of the Tetris board, measured from
      the bottom of the board to the highest occupied cell.
  \item \emph{Holes}: The number of unoccupied cells that have at least one
      occupied cell above them.
  \item \emph{Connected Holes}: Same as \emph{Holes}, except vertically
      connected unoccupied cells only count as one connected hole.
  \item \emph{Removed Lines}: The number of lines that were cleared in the last
      step to get to the current board.
  \item \emph{Altitude Difference}: The difference between the highest occupied
      cell and the lowest free cell that is directly reachable from the top.
  \item \emph{Maximum Well Depth}: The depth of the deepest well on the board.
      A well is a vertical group of unoccupied cells with a width of one,
      reachable from the top and with other filled cells on both sides.
  \setcounter{savedcounter}{\theenumi}
\end{enumerate}

This coursework extends the board-rating function by adding the additional six
criteria used by Mandl \cite{Mandl2005}:

\begin{enumerate}
  \setcounter{enumi}{\thesavedcounter}
  \item \emph{Sum of all Wells}: Sum of the depths of all the wells on the board.
  \item \emph{Landing Height}: The height at which the last Tetramino was placed.
  \item \emph{Blocks}: The total number of occupied cells currently on the board.
  \item \emph{Weighted Blocks}: Same as \emph{Blocks} above, but each occupied
      cell at height $n$ from the bottom of the board counts for $n$.
  \item \emph{Row Transitions}: The sum of all horizontal occupied/unoccupied
      transitions across the board.
      The area outside the board counts as occupied.
  \item \emph{Column Transitions}: The sum of all vertical occupied/unoccupied
      transitions across the board.
      The area above the board counts as unoccupied, and the area below the
      board counts as occupied.
\end{enumerate}

Two additional board rating functions were also implemented.
The previous coursework exclusively used a linear rating function $R_l(b)$
shown in Equation \ref{LinearRating} below, but here we have added the two
other rating functions described in Mandl's paper.
These are the exponential function $R_e(b)$ and the exponential-displacement
function $R_d(b)$.


\begin{equation}
\label{LinearRating}
  R_l(b) = \sum^6_{i=1} w_ir_i(b)
\end{equation}

\begin{equation}
\label{ExponentialRating}
  R_e(b) = \sum^6_{i=1} w_ir_i(b)^{e_i}
\end{equation}

\begin{equation}
\label{DisplacementRating}
  R_d(b) = \sum^6_{i=1} w_i \lvert r_i(b) - d_i \rvert ^{e_i}
\end{equation}

These functions add some more genes to the individual - the set of exponents
$e$ and the set of displacements $d$.
Both these sets contain real-valued numbers and are the same size as the set of
weights $w$ used in the original individual.
When initially seeding the player population with individuals we chose integer
values for the weights randomly from the range $-1000 \cdots 1000$.
Similarly we choose random real numbers for the exponents from the range
$-2.0 \cdots 2.0$, and displacements from $-10.0 \cdots 10.0$.

\subsection{Random numbers}

The previous coursework generated pseudo-random numbers by calling the
\emph{rand\_r()} C library function.
The implementation of \emph{rand\_r()} suggested in the C99 specification
\cite{C99} uses a 16-bit seed and is well known for its poor statistical
properties and small cycle especially in the low-order bits.
The implementation in the GNU C library is slightly better, as it uses a 32-bit
seed, but still suffers from the same problems.

We make use of random numbers in almost every part of this project, and the
reliability of our results depends on these numbers being unpredictable and with
a very high cycle.
Because we generate so many random numbers it is also advantageous for the
psuedo-random number generator to be fast (\emph{rand\_r()} is very slow).

The Mersenne Twister \cite{matsumoto1998mersenne} generator has a cycle of
$2^{19937}$ (compared to \emph{rand\_r()}'s $2^{32}$) and is generally regarded as
being the most suitable general-purpose generator.
However we chose to use a Lagged Fibonacci \cite{brent1992uniform} generator
with parameters $p=607$, $q=273$.
This generator has a faster execution time than Mersenne Twister and repeats in
a cycle of $2^{32000}$, however this comes at the cost of increased state (4.7kb
instead of 2.5kb).

The implementations from the Boost Random Number
Library\footnote{\url{http://www.boost.org}} were used.

\subsection{Population diversity}

In a long running evolution it is possible for the individuals to become
genetically very similar to each other as the least fit ones are eliminated and
the fittest reproduce amongst themselves.
Diversity in the population is required to ensure that the algorithm does not
``home in'' on a local minimum without any chance of jumping away.

We would like to measure the diversity of the genes within chromosomes (weights,
exponents and displacements) in our Tetris player population.
This measure can then be used to decide whether it is necessary to introduce
fresh, random individuals into the population.

Nsakanda et al.\ \cite{nsakanda2007ensuring} provide an overview of the common
ways of measuring population diversity in genetic algorithms, including the
entropy measure suggested by Grenfenstette \cite{grefenstette1987incorporating}.
Burke et al.\ \cite{burke2003advanced} compare a selection of more advanced
population diversity measures.

For this project we decided to settle on a more straightforward way of measuring
diversity: standard deviation.
If we take the value of the $j^\textrm{th}$ gene from individual $i$ to be
$g_{ij}$, then the diversity of the population can be measured as:

\begin{equation}
  D = \sum_{i=1}^{n_p} \frac{\sqrt{\frac{\sum_{j=1}^{n_g} (g_{ij} - \overline{g_i})^2 }{n_g}} }{n_p}
\end{equation}

Where $n_p$ is the number of individuals in the population, and $n_g$ is the
number of genes composing each individual.
The diversity was of course measured seperately for each chromosome as the
genes in different chromosomes have different precisions and ranges and would be
incompatible when measured together.

\section{The Problem}

\subsection{Overview}

Tetris has a large random component.
Even a good strategy can be defeated quickly by an unfortunate sequence of blocks.

Can we evolve a sequence of tetraminos that trains players to be good on a
variety of random sequences?

Coevolution has been used to improve game players before.
Chinese Chess \cite{ong2007discovering}
Checkers \cite{chellapilla1999evolving}
Othello \cite{moriarty1995discovering}

\subsection{The new individual}

Types of representation.
-a finite state automaton

Crossover.

Mutation.

\section{Results}

\section{Evaluation}

\bibliography{../COMP6026}

\end{document}
